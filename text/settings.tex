


%%%%% CHOOSE YOUR LINE SPACING HERE
% This is the official option.  Use it for your submission copy and library copy:
%\setlength{\textbaselineskip}{22pt plus2pt}
% This is closer spacing (about 1.5-spaced) that you might prefer for your personal copies:
\setlength{\textbaselineskip}{18pt plus2pt minus1pt}

% You can set the spacing here for the roman-numbered pages (acknowledgements, table of contents, etc.)
\setlength{\frontmatterbaselineskip}{17pt plus1pt minus1pt}

% Leave this line alone; it gets things started for the real document.
\setlength{\baselineskip}{\textbaselineskip}


%%%%% CHOOSE YOUR SECTION NUMBERING DEPTH HERE
% You have two choices.  First, how far down are sections numbered?  (Below that, they're named but
% don't get numbers.)  Second, what level of section appears in the table of contents?  These don't have
% to match: you can have numbered sections that don't show up in the ToC, or unnumbered sections that
% do.  Throughout, 0 = chapter; 1 = section; 2 = subsection; 3 = subsubsection, 4 = paragraph...

% The level that gets a number:
\setcounter{secnumdepth}{2}
% The level that shows up in the ToC:
\setcounter{tocdepth}{2}


%%%%% ABSTRACT SEPARATE
% This is used to create the separate, one-page abstract that you are required to hand into the Exam
% Schools.  You can comment it out to generate a PDF for printing or whatnot.
\begin{abstractseparate}
	Your abstract text goes here.  Check your departmental regulations, but generally this should be less than 300 words.  See the beginning of Chapter~\ref{ch:2-litreview} for more.

Lorem ipsum dolor sit amet, consectetur adipiscing elit. Pellentesque sit amet nibh volutpat, scelerisque nibh a, vehicula neque. Integer placerat nulla massa, et vestibulum velit dignissim id. Ut eget nisi elementum, consectetur nibh in, condimentum velit. Quisque sodales dui ut tempus mattis. Duis malesuada arcu at ligula egestas egestas. Phasellus interdum odio at sapien fringilla scelerisque. Mauris sagittis eleifend sapien, sit amet laoreet felis mollis quis. Pellentesque dui ante, finibus eget blandit sit amet, tincidunt eu neque. Vivamus rutrum dapibus ligula, ut imperdiet lectus tincidunt ac. Pellentesque ac lorem sed diam egestas lobortis.

Suspendisse leo purus, efficitur mattis urna a, maximus molestie nisl. Aenean porta semper tortor a vestibulum. Suspendisse viverra facilisis lorem, non pretium erat lacinia a. Vestibulum tempus, quam vitae placerat porta, magna risus euismod purus, in viverra lorem dui at metus. Sed ac sollicitudin nunc. In maximus ipsum nunc, placerat maximus tortor gravida varius. Suspendisse pretium, lorem at porttitor rhoncus, nulla urna condimentum tortor, sed suscipit nisi metus ac risus.

Aenean sit amet enim quis lorem tristique commodo vitae ut lorem. Duis vel tincidunt lacus. Sed massa velit, lacinia sed posuere vitae, malesuada vel ante. Praesent a rhoncus leo. Etiam sed rutrum enim. Pellentesque lobortis elementum augue, at suscipit justo malesuada at. Lorem ipsum dolor sit amet, consectetur adipiscing elit. Praesent rhoncus convallis ex. Etiam commodo nunc ex, non consequat diam consectetur ut. Pellentesque vitae est nec enim interdum dapibus. Donec dapibus purus ipsum, eget tincidunt ex gravida eget. Donec luctus nisi eu fringilla mollis. Donec eget lobortis diam.

Suspendisse finibus placerat dolor. Etiam ornare elementum ex ut vehicula. Donec accumsan mattis erat. Quisque cursus fringilla diam, eget placerat neque bibendum eu. Ut faucibus dui vitae dolor porta, at elementum ipsum semper. Sed ultrices dui non arcu pellentesque placerat. Etiam posuere malesuada turpis, nec malesuada tellus malesuada. % Create an abstract.tex file in the 'text' folder for your abstract.
\end{abstractseparate}


% JEM: Pages are roman numbered from here, though page numbers are invisible until ToC.  This is in
% keeping with most typesetting conventions.
\begin{romanpages}

% Title page is created here
\maketitle

%%%%% DEDICATION -- If you'd like one, un-comment the following.
%\begin{dedication}
%This thesis is dedicated to\\
%someone\\
%for some special reason\\
%\end{dedication}

%%%%% ACKNOWLEDGEMENTS -- Nothing to do here except comment out if you don't want it.
\begin{acknowledgements}
 	\subsection*{Personal}

This is where you thank your advisor, colleagues, and family and friends.

Lorem ipsum dolor sit amet, consectetur adipiscing elit. Vestibulum feugiat et est at accumsan. Praesent sed elit mattis, congue mi sed, porta ipsum. In non ullamcorper lacus. Quisque volutpat tempus ligula ac ultricies. Nam sed erat feugiat, elementum dolor sed, elementum neque. Aliquam eu iaculis est, a sollicitudin augue. Cras id lorem vel purus posuere tempor. Proin tincidunt, sapien non dictum aliquam, ex odio ornare mauris, ultrices viverra nisi magna in lacus. Fusce aliquet molestie massa, ut fringilla purus rutrum consectetur. Nam non nunc tincidunt, rutrum dui sit amet, ornare nunc. Donec cursus tortor vel odio molestie dignissim. Vivamus id mi erat. Duis porttitor diam tempor rutrum porttitor. Lorem ipsum dolor sit amet, consectetur adipiscing elit. Sed condimentum venenatis consectetur. Lorem ipsum dolor sit amet, consectetur adipiscing elit.

Aenean sit amet lectus nec tellus viverra ultrices vitae commodo nunc. Mauris at maximus arcu. Aliquam varius congue orci et ultrices. In non ipsum vel est scelerisque efficitur in at augue. Nullam rhoncus orci velit. Duis ultricies accumsan feugiat. Etiam consectetur ornare velit et eleifend.

Suspendisse sed enim lacinia, pharetra neque ac, ultricies urna. Phasellus sit amet cursus purus. Quisque non odio libero. Etiam iaculis odio a ex volutpat, eget pulvinar augue mollis. Mauris nibh lorem, mollis quis semper quis, consequat nec metus. Etiam dolor mi, cursus a ipsum aliquam, eleifend venenatis ipsum. Maecenas tempus, nibh eget scelerisque feugiat, leo nibh lobortis diam, id laoreet purus dolor eu mauris. Pellentesque habitant morbi tristique senectus et netus et malesuada fames ac turpis egestas. Nulla eget tortor eu arcu sagittis euismod fermentum id neque. In sit amet justo ligula. Donec rutrum ex a aliquet egestas.

\subsection*{Institutional}

If you want to separate out your thanks for funding and institutional support, I don't think there's any rule against it.  Of course, you could also just remove the subsections and do one big traditional acknowledgement section.

Lorem ipsum dolor sit amet, consectetur adipiscing elit. Ut luctus tempor ex at pretium. Sed varius, mauris at dapibus lobortis, elit purus tempor neque, facilisis sollicitudin felis nunc a urna. Morbi mattis ante non augue blandit pulvinar. Quisque nec euismod mauris. Nulla et tellus eu nibh auctor malesuada quis imperdiet quam. Sed eget tincidunt velit. Cras molestie sem ipsum, at faucibus quam mattis vel. Quisque vel placerat orci, id tempor urna. Vivamus mollis, neque in aliquam consequat, dui sem volutpat lorem, sit amet tempor ipsum felis eget ante. Integer lacinia nulla vitae felis vulputate, at tincidunt ligula maximus. Aenean venenatis dolor ante, euismod ultrices nibh mollis ac. Ut malesuada aliquam urna, ac interdum magna malesuada posuere.
\end{acknowledgements}

%%%%% ABSTRACT -- Nothing to do here except comment out if you don't want it.
\begin{abstract}
	Your abstract text goes here.  Check your departmental regulations, but generally this should be less than 300 words.  See the beginning of Chapter~\ref{ch:2-litreview} for more.

Lorem ipsum dolor sit amet, consectetur adipiscing elit. Pellentesque sit amet nibh volutpat, scelerisque nibh a, vehicula neque. Integer placerat nulla massa, et vestibulum velit dignissim id. Ut eget nisi elementum, consectetur nibh in, condimentum velit. Quisque sodales dui ut tempus mattis. Duis malesuada arcu at ligula egestas egestas. Phasellus interdum odio at sapien fringilla scelerisque. Mauris sagittis eleifend sapien, sit amet laoreet felis mollis quis. Pellentesque dui ante, finibus eget blandit sit amet, tincidunt eu neque. Vivamus rutrum dapibus ligula, ut imperdiet lectus tincidunt ac. Pellentesque ac lorem sed diam egestas lobortis.

Suspendisse leo purus, efficitur mattis urna a, maximus molestie nisl. Aenean porta semper tortor a vestibulum. Suspendisse viverra facilisis lorem, non pretium erat lacinia a. Vestibulum tempus, quam vitae placerat porta, magna risus euismod purus, in viverra lorem dui at metus. Sed ac sollicitudin nunc. In maximus ipsum nunc, placerat maximus tortor gravida varius. Suspendisse pretium, lorem at porttitor rhoncus, nulla urna condimentum tortor, sed suscipit nisi metus ac risus.

Aenean sit amet enim quis lorem tristique commodo vitae ut lorem. Duis vel tincidunt lacus. Sed massa velit, lacinia sed posuere vitae, malesuada vel ante. Praesent a rhoncus leo. Etiam sed rutrum enim. Pellentesque lobortis elementum augue, at suscipit justo malesuada at. Lorem ipsum dolor sit amet, consectetur adipiscing elit. Praesent rhoncus convallis ex. Etiam commodo nunc ex, non consequat diam consectetur ut. Pellentesque vitae est nec enim interdum dapibus. Donec dapibus purus ipsum, eget tincidunt ex gravida eget. Donec luctus nisi eu fringilla mollis. Donec eget lobortis diam.

Suspendisse finibus placerat dolor. Etiam ornare elementum ex ut vehicula. Donec accumsan mattis erat. Quisque cursus fringilla diam, eget placerat neque bibendum eu. Ut faucibus dui vitae dolor porta, at elementum ipsum semper. Sed ultrices dui non arcu pellentesque placerat. Etiam posuere malesuada turpis, nec malesuada tellus malesuada.
\end{abstract}

%%%%% MINI TABLES
% This lays the groundwork for per-chapter, mini tables of contents.  Comment the following line
% (and remove \minitoc from the chapter files) if you don't want this.  Un-comment either of the
% next two lines if you want a per-chapter list of figures or tables.
\dominitoc % include a mini table of contents
%\dominilof  % include a mini list of figures
%\dominilot  % include a mini list of tables

% This aligns the bottom of the text of each page.  It generally makes things look better.
\flushbottom

% This is where the whole-document ToC appears:
\tableofcontents

\listoffigures
	\mtcaddchapter
% \mtcaddchapter is needed when adding a non-chapter (but chapter-like) entity to avoid confusing minitoc

% Uncomment to generate a list of tables:
%\listoftables
%	\mtcaddchapter

%%%%% LIST OF ABBREVIATIONS
% This example includes a list of abbreviations.  Look at text/abbreviations.tex to see how that file is
% formatted.  The template can handle any kind of list though, so this might be a good place for a
% glossary, etc.
%\include{text/abbreviations}
\newglossaryentry{utc}{name=UTC, description={Coordinated Universal Time}}
\newglossaryentry{adt}{name=ADT, description={Atlantic Daylight Time}}
\newglossaryentry{est}{name=EST, description={Eastern Standard Time}}
% You can add abbreviations into glossary.tex file.
\printglossary[title={List of Abbreviations}]

% The Roman pages, like the Roman Empire, must come to its inevitable close.
\end{romanpages}
